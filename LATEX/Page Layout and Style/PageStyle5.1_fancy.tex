\documentclass[a4paper,twoside]{book}
\usepackage{fancyheadings}

\pagestyle{fancy}
\renewcommand{\chaptermark}[1]{\markboth{\thechapter. #1}{}}
\renewcommand{\sectionmark}[1]{\markright{\thesection. #1}}

\rhead[\textbf{\leftmark}]{\textbf{\thepage}} % optional argument becomes active only if twoside printing is opted
\lhead[\textbf{\thepage}]{\textbf{\rightmark}}
\rfoot[]{\textbf{Dilip Datta}}
\lfoot[\textbf{Engineering Mechanics}]{}
\cfoot[]{}
\renewcommand{\headrulewidth}{0.15mm}
\renewcommand{\footrulewidth}{0.15mm}
\addtolength{\headwidth}{\marginparsep}
\addtolength{\headwidth}{\marginparwidth}
%
\begin{document}
 ...
 \chapter*{Preface}
 \huge 
 The page style fancy, 
 defined in the fancyheadings package, allows very elegant customization of the running header 
 and footer of a document. 
 The package provides three types of headers as well as footers, through which a header/footer can be made page-wise left, center or right aligned, 
 or even multiple pieces of headers and footers can be used. 
 The commands for such headers and footers are shown in Table 5.7, where podd and peven are the contents of the headers/footers on odd and
 
 \chapter{First Chapter}
 %\thispagestyle{empty} %to locally appy empty style in a particular page
     The beginnig of first chapter.
         \section{First Section}
         The start of section one.
             \subsection{First subsection}
             The start of sub-section one.
                 \subsubsection{First sub-sub-section}
                     The start of sub-sub-section one. Sub-sub-sections are {\bf\Huge not} numbered in case of {\bf\huge article} class.
                     \newpage
                     ....
                     \newpage
                     =====
 \chapter{Running Header and Footer}
 \bfseries\large
 By default, /thispagestyle{plain} is issued by the document-classes of article, book, and report to the 
 /maketitle command and the first page of major sectioning commands,like /part{ } or /chapter{ }. 
 To suppress the effect of /thispagestyle{plain} on these pages, /thispagestyle{ } with an appropriate
  page style may be used just after the /maketitle, /part{ } or /chapter{ } command, 
  e.g., /thispagestyle{empty} may be used to suppress the default page numbering on these pages.
     \section{1st Section}
     .....
     \newpage
     .....
     \newpage
     \section{2nd Section}
     .........
     \newpage
     ....
     \chapter{3rd Chapter}
     ....
     \newpage
        ---
\end{document}