\documentclass[a4paper]{article}
% 1pt = .3515mm
% values gotten from \layout command defined in the layout package.
\voffset = 0pt %vertical coordinates of the reference point,
\hoffset = 0pt %the horizontal coordinates of the reference point,
%
\paperwidth = 210mm % a4 
\paperheight = 297mm % a4
\textheight = 50mm %Height of main texts without header and footer. Not default
\textwidth = 100mm %Width of main texts without marginal notes. Not default
%Header related parameters , set with def values 
\topmargin = 20pt %Extra vertical space above the header.
\headheight = 12pt %Height of header
\headsep = 25pt %Vertical gap between the header and the first line of the main texts.
%Footer related
\footskip = 30pt %Vertical gap between the last line of the main texts and the footer.
%Multiple Column related
% \columnseprule = 1mm %default value is 0mm i.e, invisible
% \columnsep = 10mm % default = 3.5 mm , specifies the gap between two columns
% \multicolsep = 20mm % default = 4.5 mm , controls the vertical gap before and after the environment
% \columnwidth is automatically calculated \textwidth and \columnsep
%Marginal Notes
% \marginparwidth = 35pt %the width of a marginal note , def = 35pt
% \marginparsep = 10pt %the horizontal gap between the main texts of a page and a marginal note, def = 10pt
% \marginparpush = 7pt %the vertical gap between two successive marginal notes, def = 7pt


\begin{document}
\centering %center alignment

Text Area is Shown Beelow
\rule{\textwidth}{\textheight}

Line Width
\rule{\linewidth}{1pt} %creates a horizontal rule (line) that spans the entire width of the line.
\\[20pt]
Text Width
\rule{\textwidth}{1pt}


\newpage
\Large
All the standard papers listed in Table 5.1 have some fixed values for the commands controlling the dimensional parameters of a page layout. 
Even for the same paper, the value of a command may vary with the size of fonts as well as with the type of printing (single-side or both-side). 
However, the values of these commands are independent of the three standard document-classes of article, book, and report. For a particular setting, 
the page layout similar to the one shown in Fig. 5.1, along with the values of some parameter controlling commands, can be obtained through the 
/layout command defined in the layout package
\end{document}






