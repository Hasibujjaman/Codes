%2003001
\documentclass[a4paper]{article}
\usepackage{amsmath,amssymb}

\begin{document}
If $f(x, y)$ is a function, where f partially depends on $x$ and $y$ and if we differentiate $f$ with respect to $x$ and $y$ then the derivatives are called the partial derivative of f. The formula for partial derivative of $f$ with respect to $x$ taking $y$ as a constant is given by:

\begin{equation*} % for single equation
    f_x = \frac{\partial{f}}{\partial{x}} = \lim_{h\to 0}\frac{f(x+h,y) - f(x,y)}{h}
\end{equation*}

And partial derivative of function f with respect y keeping x constant,\\we get;
\begin{equation*} % for single equation
    f_y = \frac{\partial{f}}{\partial{y}} = \lim_{h\to 0}\frac{f(x,y+h) - f(x,y)}{h}
\end{equation*}

Consider the following function: f(x,y) = x2y. Partial derivatives of this function are:
\begin{eqnarray*}
f_x &=& \frac{\partial{f}}{\partial{x}}\\
    &=& \frac{\partial}{\partial{x}}(x^2y)\\
    &=& 2xy
\end{eqnarray*}
\begin{eqnarray*}
    f_xy &=& \frac{\partial{f}}{\partial{y}}\\
        &=& \frac{\partial}{\partial{x}}(x^2y)\\
        &=& x^2
\end{eqnarray*}
\end{document}