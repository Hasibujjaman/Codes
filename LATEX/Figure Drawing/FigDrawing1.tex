%drwaing lines and vectors
\documentclass[a4paper]{article}
\usepackage{graphicx}
\usepackage{subfigure}
\begin{document}
\noindent
%\begin{picture}(lx,ly)(x0,y0),where(x0,y0) is the lower left coordinate of the window, and lx and ly are its lengths along the x- and y- axes, respectively
    This is a line before the figure\\
    \setlength{\unitlength}{1mm}
    \begin{picture}(60,30)(0,0)
        \thinlines %def
        \put(0,0){\line(0,1){30}} % \put(x,y){fcmd} command to start the figure at (x,y).
        \put(5,0){\line(1,0){55}}
        \put(45,25){\line(1,-1){15}}
    \end{picture}\\
    End of figure\\[2cm]

    \setlength{\unitlength}{1mm}
    This is a line before the figure\\
    \begin{figure}[!htb]
        \centering
        \caption{y = x}
        \label{line_1}
        \begin{picture}(60,30)(-30,-15)
            \thicklines
            \put(0,0){\vector(1,0){30}}
            \put(0,0){\vector(-1,0){30}}
            \put(0,0){\vector(0,1){15}}
            \put(0,0){\vector(0,-1){15}}
            \put(-30,-15){\line(1,0){60}}
            \put(-30,-15){\line(0,1){30}}
            \put(-30,15){\line(1,0){60}}
            \put(30,15){\line(0,-1){30}}
            \put(0,0){\line(1,1){10}} % \put(x,y){fcmd} command to start the figure at (x,y).
        \end{picture} 
    \end{figure}

    Here ends a figure\\[10mm]

    \begin{picture}(120,60)(0,0)
        \thicklines
        \put(5,5){\vector(1,2){20}}
        \put(75,15){\vector(-1,0){50}}
        \put(115,55){\vector(-1,-2){25}}
        \end{picture}

    Here ends a figure\\[10mm]

    \setlength{\unitlength}{2mm}
    \begin{picture}(60,30)(0,0)
        \linethickness{2mm}
        \multiput(5,5)(5,10){3}{\line(1,0){20}} %The \multiput(x,y)(delx,dely){n}{fcmd} command can also be used for draw- ing the same figure n times, starting the first one at (x,y) and incrementing (x,y) each time by (delx,dely) for the subsequent figures 
        \multiput(40,5)(7,2){3}{\vector(1,2){10}}
    \end{picture}

    Here ends a figure\\[10mm]

\end{document}