\documentclass[a4paper,openany]{report}
%Texual materials like algorithms and codes can be forced through a floating environment to appear together on a single page like table and figures
\usepackage{float} 
%the following three commands need to be in this order
\floatstyle{ruled} % plain/boxed/ruled
\newfloat{algorithm}{hbt}{alg}[section] %  newfloat{afloat}{vpos}{extn}[unit] 
\floatname{algorithm}{Algorithm} % \floatname{afloat} {clabel} for redefining float caption label

\floatstyle{boxed} 
\newfloat{program}{hbt}{prg}[chapter] 
\floatname{program}{Program}
%
\begin{document}
…
\chapter{Intro to Algorithms}
    \section{Finding the maximum}
    The main steps … Algorithm~\ref{algo:max}.
    \begin{algorithm}
        \caption{Maximum of $n$ data points.}
        \label{algo:max}
        \begin{enumerate}
            \item Read the number of data points $n$.
            \item Read the data point $a_i$;
            $i=1$ to $n$.
            \item Set {\it max} $=a_1$.
            ...
        \end{enumerate}
    \end{algorithm}
%
    Algorithm~\ref{algo:max} is coded in the {\tt C} computer … Program~\ref{prog:max}.
    \begin{program}
        \caption{Maximum of $n$ data points.}
        \label{prog:max}
        \begin{verbatim}
#include<stdio.h>
#include<math.h>
int main()
{
int n, a[101];
int i, max;
printf("Number of points = ");
...
return(0);
}
    \end{verbatim}
    \end{program}
    …
\end{document}