\documentclass[twocolumn]{article}
%
\usepackage{multicol} % for multicols environment
%
\columnseprule = 1mm %default value is 0mm i.e, invisible
\columnsep = 10mm % default = 3.5 mm , specifies the gap between two columns
\multicolsep = 20mm % default = 4.5 mm , controls the vertical gap before and after the envi- ronment
%\textwidth = 50mm
% \columnwidth is automatically calculated \textwidth and \columnsep
%
\begin{document}
	This is an example where a ... fklkdsaf dlkfals kdlksfla kdmskfl
	%
	\begin{multicols}{3}
		\raggedright
		This is a three-column paragraph. ... klfkdsflksdkfsd klkksfkls kflafkdfkl sdlkflkfkldfkdfkdklasf fjadkkdfjlk djfkdflkd sajflkjds End AAAAAAAAA aaaaaaa BBBBBBB \columnbreak
		
		 \large This a new flksa kdsflsakd flksaf dsflksak fkdsflkas kfklas fdnkjfas dddddd dnkfasd fdjsnfjks fdsnjkfnsa ndjfans
	\end{multicols}
	%
	In this example, all the three ...\\
	\Large\bfseries
	In a multi-column page, the /newpage and /pagebreak commands start a new column instead of a new page. In that case, a new page can be started using /clearpage or /cleardoublepage only.
	\newpage
	/newpage or /pagebreak calls new column.
	\pagebreak 
	\normalfont

	New column
	\newpage
	new column
	\pagebreak
	new column
	\newpage
	New column
	\clearpage
	New page
	\cleardoublepage
	New new page
 \end{document}