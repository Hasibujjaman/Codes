%A math-mode environment, like equation(for single equation only) or enqarray, prints a mathematical expression in a new line.
%A mathematical notation or expression, say amath, can be inserted in text mode evironment as $amath$, \(amath\), or \begin{math} amath \end{math}.
\documentclass[a4paper]{article}
\usepackage{amsmath,amssymb}

\begin{document}
    \centering
    {\Large\bf Brackets:}\\
    The distributive property states that $a(b+c)=ab+ac$, for all $a, b, c \in \mathbb{R}$. \\[6pt]
    The equivalence class of $a$ is $[a]$.\\[6pt]
    The set $A$ is defined to be $\{1, 2, 3\}$.\\[6pt]
    The movie ticket costs $\$11.50$. 

    $$2\left(\frac{1}{x^2-1}\right)$$
    $$2\left[\frac{1}{x^2-1}\right]$$
    $$2\left\{\frac{1}{x^2-1}\right\}$$
    $$2\left \langle      \frac{1}{x^2-1}\right \rangle      $$
    $$2\left | \frac{1}{x^2-1}\right |      $$
    $$\left.\frac{dy}{dx}\right|_{x=1}$$
    $$\left( \frac{1}{1+\left(\frac{1}{1+x}\right)} \right)$$
    \newpage
 
    %%%%%%%%%%%%%%%%%%%%%%%%%%%%%%%%%%%%%%%%%%%%%%%%%%%%%%%%%%%


    %Some math mode environment
    {\Large\bf Mathematical notation:}\\
    \begin{equation} % for single equation
         %1.prime
         p'
    \end{equation}

    \begin{align}
        \text{The whole environment is in mathmode}\nonumber \\
        %2.dots
        \dot{x}, \ddot{x}, \dddot{x}, \ddddot{x} x_i, x^2 \\
        %3.Single sub-/super-script
        x_{ij}, x^{2k}\\
        %4.Multiple sub-/super-scripts
        x^{2k}_{ij} or x_{ij}^{2k}\\
        %5.Summation
        \sum, \sum_{i=1}^{20} \label{sum1}\\
        %6.Product
        \prod, \prod_{i=1}^{i=20}\\
        %7.Integration
        \int x^2\,dx,\int_a^b xy\,dx \\
        %8.Mutiple Integration
        \iint\limits_s,\iiint\limits_v, \iiiint \\
        %9.Set of integrations
        \idotsint \\
        %10.Cyclic integration
        \oint \\
        %11.Fraction
        \frac{x}{y} \\
        %12.Derivative
        \nabla{f}, \frac{dx}{dy} \\
        %13.Partial derivative
        \frac{\partial{y}}{\partial{x}}\\
        %14.Root
        \sqrt{x}, \sqrt[5]{xyz} \\
        %15.Limit
        \lim_{x\to 0}, \underset{x\to 0}{\lim}\\
        %16.Exists/not exists
        \exists,\nexists \\
        %17.Modes
        \mod{n^2},\bmod{n^2},\pmod{n^2},\pod{n^2} \\
        %18.Binomial expression
        \binom{n}{k}
    \end{align}
    Eqn. \ref{sum1} is for Summation symbols.
    \newpage
    %%%%%%%%%%%%%%%%%%%%%%%%%%%%%%%%%%%%%%%%%%%%%%%
    {\Large\bf Basic operators:}\\
   \begin{eqnarray*}
    \leq \\
    \geq \\ 
    \ll \\
    \gg \\
    \subset \\
    \subseteq \\
    \in \\
    \not\in \\
    \equiv \\
    \sim \\
    \approx \\
    \neq \\
    \propto \\
    \not< \\
    \not> 
   \end{eqnarray*}

\end{document}